\documentclass[11pt]{beamer}

% Hey Nick!
% Percent sign is a comment
% \begin{frame} \end{frame} is a page on the presentation
% \section denotes a new section, \subsection new subsection, \subsubsection, etc (used for Table of Contents)
% \pause Things after \pause will only show on pressing space, return, or page down, etc.
% See: https://en.wikibooks.org/wiki/LaTeX/Presentations for more information :-)

\usetheme{Berlin}
\usecolortheme{beaver}
\beamertemplatenavigationsymbolsempty

%Gummi|065|=)
\title[Bitmap Indexes for Large Datasets]{\textbf{Using Bitmap Index for Interactive Exploration of Large Datasets}}
\subtitle{INFO3504 Presentation}
\author{Nick Armstrong \\
        Daniel Collis}
\institute{University of Sydney}
\date{}
\subject{Computer Science}

\begin{document}

% Front page
\frame{\titlepage}

% Table of Contents
\begin{frame}
	\frametitle{Table of Contents}
	\tableofcontents
\end{frame}

\section{Introduction}

\subsection[Bitmap Indexes]{Bitmap Indexes}

\begin{frame}
	\frametitle{Bitmap Indexes}
	
	Bitmap Indexes:
	\begin{itemize}
		\pause
		\item Are an old idea
		\pause
		\item Are efficient for big data and data warehouse solutions with read-only records
		\pause
		\item Do not reorder data, unlike some other algorithms
	\end{itemize}
\end{frame}

\subsubsection[Bitmap Indexes]{What Are They?}
\begin{frame}
	\frametitle{What Are They?}
	
	For those who weren't in last week's lecture, they \pause are a way to represent records and fields as a grid (essentially).
	
	\pause
	\begin{exampleblock}{Just imagine...}
		Imagine a table, with a bunch of records ($R_x$) and an enumeration ($A_x = \{1, 2, 3\}$):
		
		\pause
		\begin{tabular}{r|c|c|c|c}
				& \textbf{Value} & $A_1$ & $A_2$ & $A_3$ \\ \hline
			$R_1$ & $3$	& 	& 	& \\
			$R_2$ & $1$	& 	& 	& \\
			$R_3$ & $1$	& 	& 	& \\
			$R_4$ & $2$	& 	& 	& \\
			$R_5$ & $3$	& 	& 	& 
		\end{tabular}
	\end{exampleblock}
\end{frame}

\begin{frame}
	\frametitle{What Are They?}
	
	For those who weren't in last week's lecture, they are a way to represent records and fields as a grid (essentially).
	
	\begin{exampleblock}{Just imagine...}
		Imagine a table, with a bunch of records ($R_x$) and an enumeration ($A_x = \{1, 2, 3\}$):
		
		\begin{tabular}{r|c|c|c|c}
				& \textbf{Value} & $A_1$ & $A_2$ & $A_3$ \\ \hline
			$R_1$ & $3$	& $0$	& $0$	& $1$ \\
			$R_2$ & $1$	& $1$	& $0$	& $0$ \\
			$R_3$ & $1$	& $1$	& $0$	& $0$ \\
			$R_4$ & $2$	& $0$	& $1$	& $0$ \\
			$R_5$ & $3$	& $0$	& $0$	& $1$ 
		\end{tabular}
	\end{exampleblock}
\end{frame}

\begin{frame}
	\frametitle{What Are They?}
	
	The example table mentioned before can become really really big with multiple attributes and multiple rows.
	\pause
	\begin{exampleblock}{Lot's of attributes!}
		It could look something like this:
		
		\begin{columns}[c]
			\column{.4\textwidth}
			\begin{tabular}{r|c|c|c|c}
					& \textbf{Value} & $A_1$ & $A_2$ & $A_3$ \\ \hline
				$R_1$ & $3$	& $0$	& $0$	& $1$ \\
				$R_2$ & $1$	& $1$	& $0$	& $0$ \\
				$R_3$ & $1$	& $1$	& $0$	& $0$ \\
				$R_4$ & $2$	& $0$	& $1$	& $0$ \\
				$R_5$ & $3$	& $0$	& $0$	& $1$ 
			\end{tabular}
			\column{.4\textwidth}
			\begin{tabular}{r|c|c|c|c}
					& \textbf{Value} & $B_1$ & $B_2$ & $B_3$ \\ \hline
				$R_1$ & $1$	& $1$	& $0$	& $0$ \\
				$R_2$ & $1$	& $1$	& $0$	& $0$ \\
				$R_3$ & $3$	& $0$	& $0$	& $1$ \\
				$R_4$ & $2$	& $0$	& $1$	& $0$ \\
				$R_5$ & $3$	& $0$	& $0$	& $1$ 
			\end{tabular}
		\end{columns}	
	\end{exampleblock}
	
	\pause
	How to determine if a record matches multiple attributes? Use a bitwise-AND!
	
	\pause
	This is quite a quick and efficient operation.
\end{frame}

\subsection[Bitmap Compression]{Bitmap Compression}
\begin{frame}
	\frametitle{Bitmap Compression}
	
	Bitmap Indexes are great, efficient, and all. But, they can become very large!
	\pause
	The researchers of this paper had two data samples:
	\begin{itemize}
		\pause
		\item $1.6$ GiB
		\pause
		\item $30$ GiB
	\end{itemize}
	
	\pause
	Storing these in a typical database with a \texttt{Bitmap Index} took $3\times$ the size, similar to when using a \texttt{B-Tree Index} \pause $=$ Not ideal!
\end{frame}

\begin{frame}
	\frametitle{Bitmap Compression}
\end{frame}

\end{document}
